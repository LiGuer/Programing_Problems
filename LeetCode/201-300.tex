
* 204. Count Primes

	\Code{C++}
		class Solution {
		public:
			int countPrimes(int n) {
				if(n == 0 || n == 1)
					return 0;
				
				bool f[n];
				int ans = 0;
				
				for(int i=2;i<n;i++){
					f[i] = 1;
				}
				
				for (int i = 2; i < n; i++) {
					if (f[i]){
						ans++;
						for (int j = i + i; j < n; j += i) 
							f[j] = false;
					}

				}
				return ans;
			}
		};

* 207. Course Schedule
	\Code{C++}
		class Solution {
		public:
			bool canFinish(int m, vector<vector<int>>& a) {
				int n = a.size();
				
				if(n <= 1 || m == 1)
					return true;
				
				int f[m], Fg = 0, Fg2 = 0;
				
				memset(f, 0, sizeof(int) * m);
				
				while(true){
					for(int i=0;i<n;i++){
						if(f[a[i][1]] != -1){
							f[a[i][0]]++;
						}
					}
					Fg = 0;
					
					for(int i=0;i<m;i++){
						if(f[i] == 0){
							f[i] = -1;Fg++;
						}
						else if(f[i] > 0){
							f[i] = 0;
						}
						else if(f[i] == -1){
							Fg++;
						}
					}
					

					if(Fg == Fg2)
						return false;
					else if(Fg == m){
						return true;
					}
					else Fg2 = Fg;
				}
				return true;

			}
		};

* 214 Shortest Palindrome

\Code{C++}
	class Solution {
	public:
			int fun (string s) {
					int n = s.length();
					int g[n + 1];

					memset(g, 0, sizeof(g));
					
					for (int i = 2; i <= n; i++) {
							int j = g[i - 1];

							while (j != 0 && s[j + 1] != s[i]) {
									j = g[j];
							}
							if (s[i] == s[j + 1])
									j++;

							g[i] = j;
					}



					return g[n-1];
			}


			string shortestPalindrome(string s) {
					int n = s.length();

					int m = 0;

					string s1 = s;

					reverse(s1.begin(), s1.end());

					string s2 = ' ' + s + '#' + s1;

					int maxn = fun(s2) - 1;
					

					string a = "";

					for (int i = n - 1; i > maxn; i--) {
							a += s[i];
					}

					return a + s;
			}
	};

* 233 Number of Digit One
\Problem
	(input) $n \in \mathbb N$
		求所有比$n$小(包括$n$)的正整数中,含数字$1$的个数.

\Property
	- 第$i$出现$1$的现象, 是一个周期$10^{i+1}$宽$10^i$高$1$的矩形波.
		第$i$位$1$的矩形位于整个周期的第$10^i+1 ~ 2 × 10^i$个数上,
		$
			(\mb
				\text{数位 i} & \text{周期} & \text{波峰宽度} & \text{波峰高度}
				0 & 10 & 1 & 1
				1 & 100 & 10 & 1
				2 & 1000 & 100 & 1
				\vdots & \vdots & \vdots & \vdots
				i & 10^{i+1} & 10^i & 1
			\me)
		$

		eg.
		$
			& \mb 0 & 1 & 2 & 3 & 4 & 5 & 6 & 7 & 8 & 9 \me \tag{一位$1$在第$2$个}
			& \mb 0 & 1 & ... & 10 & ... & 19 & ... & ... & ... & 99 \me \tag{二位$1$在第$11-20$个}
			& \mb 0 & 1 & ... & 100 & ... & 199 & ... & ... & ... & 999 \me \tag{三位$1$在第$101-200$个}
		$

\Algorithm
	- 将$0 ~ n$的数分解为每一位上的$1$的矩形波, 并求和波峰即可.
	- 步骤
		- 首先, 求完成的周期数, 周期数$×$波峰宽 便是第一部分数字$1$的数量.
		- 然后, 分析正在进行的周期, 求余剩下的数的长度rest
			- $rest ≥ 2 × 10^i$ 时, 波峰部分已完成, 直接加波峰宽度即可.
			- $10^i+1 ≤ rest < 2 × 10^i$ 时, 波峰部分正在进行, 加剩余数的数目即可
			- $rest < 10^i+1 $ 时, 波峰部分未开始, 不用计算.

\Code{C++}
	class Solution {
	public:
		long long countDigitOne(long long n) {
			long long ans = 0;

			long long nt = n, dim = 0;
			while (nt != 0) {
				dim++;
				nt /= 10;
			}

			for (long long i = 0; i < dim; i++) {
				// 完成的周期
				long long tmp = (n + 1) / (long long)pow(10, i + 1) * pow(10, i);

				// 正在进行的周期
				long long rest = (n + 1) % (long long)pow(10, i + 1),
					tmp2 = rest / pow(10, i);

				if (tmp2 > 1) {			// 完成一个波峰
					tmp += pow(10, i);
				}
				else if (tmp2 == 1) {	// 正在完成一个波峰
					tmp += rest % (long long)pow(10, i);
				}
										// 未完成一个波峰 (不用计算)
				ans += tmp;
			}

			return ans;
		}
	};

* 279. Perfect Squares
\Code{C++}
	class Solution {
	public:
		int numSquares(int n) {
			vector<int> S;

			int SqrtN = sqrt(n);
			for (int i = 1; i <= SqrtN; i++) {
				if (i * i <= n)
					S.push_back(i * i);
			}
			int NS = S.size();

			int f[n+1];
			for (int i = 0; i <= n; i++) {
				f[i] = 0xFFFFFFF;
			}
			f[0] = 0;

			for (int i = 0; i <= n; i++) {
				for (int j = 0; j < NS; j++) {
					if (i - S[j] >= 0) {
						f[i] = min(f[i], f[i - S[j]] + 1);
					}
				}
			}
			return f[n];
		}
	};

* 300. Longest Increasing Subsequence
	\Code{C++}
		class Solution {
		public:
			int lengthOfLIS(vector<int>& a) {
				int n = a.size();

				int f[n], ans = 0;

				for (int i = 0; i < n; i++) {
					f[i] = 0x7FFFFFF;
				}

				f[0] = a[0];

				for (int i = 1; i < n; i++) {
					if (a[i] > f[ans]) {
						ans++;
						f[ans] = a[i];
					}

					else {
						int ind = search(f, n, a[i]);
						f[ind] = a[i];
					}
				}
				return ans + 1;
			}

			int search(int* a, int n, int b) {
				int l = 0, r = n - 1, mid;
				while (l <= r) {
					mid = (l + r) >> 1;
					if (a[mid] < b) {
						l = mid + 1;
					}
					else {
						r = mid - 1;
					}
				}
				return l;
			}
		};
