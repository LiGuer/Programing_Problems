* 2 Add Two Numbers
	\Problem

	\Code{C++}
		class Solution {
		public:
			ListNode* addTwoNumbers(ListNode* l1, ListNode* l2) {
				int s = 0;
				ListNode* ans = new ListNode;
				ListNode* ansTmp = ans;
				
				while(l1 !=nullptr || l2 !=nullptr){
					ListNode* node = new ListNode;
					ansTmp->next = node;
					ansTmp = node;
					
					int tmp = 0;
					if(l1 !=nullptr && l2 !=nullptr){
						tmp = l1->val + l2->val + s;
						l1 = l1->next;
						l2 = l2->next;
					}
					else if(l1 !=nullptr){
						tmp = l1->val + s;
						l1 = l1->next;
					}
					else{
						tmp = l2->val + s;
						l2 = l2->next;
					} 
					
					s = 0;
					
					if(tmp >= 10){
						s = 1;
						tmp -= 10;
					}
					
					ansTmp->val = tmp;
				}
				
				if(s==1){
					ListNode* node = new ListNode;
					ansTmp->next = node;
					ansTmp = node;
					ansTmp->val = 1;
				}
				
				ans = ans->next;
				return ans;
			}
		};

* 5 Longest Palindromic Substring
	\Problem 
		(input) $\.a \qu; a_i \in \mathbb Z$
		$
			\max_{\.x \in \mathbb Z^{1:\dim(\.a)}} \qu& n
			s.t. \qu& n = \dim(\.x)
				& x_i = a_{s+i}  \qu; i = 1:n  \tag{子序列约束}
				& x_i = x_{n - i}  \qu; i = 1:n  \tag{回文约束}
		$

	\Algorithm
		- 动态规划
			$
				f(s,e) &= \{\mb
					f(s-1,e+1) + 2 \qu& f(s,e) > 0 \ and\  a_{s-1} = a_{e+1}
					0 \qu& other.
					\me\right.
				f(s,s) &= 1  \tag{初始易知值}
				f(s,s+1) &= 2 \qu;a_s = a_{s+1}
			$
			- $f()$: $a_{s:e}$的回文字数, 不是回文序列则为0.

	\Code{C++}
		class Solution {
		public:
			string longestPalindrome(string s) {
				int n = s.length();

				int f[n][n];

				for (int i = 0; i < n; i++) {
					for (int j = i; j < n; j++) {
						f[i][j] = 0;
					}
				}

				for (int i = 0; i < n; i++) {
					f[i][i] = 1;

					if (i != n - 1 && s[i] == s[i + 1]) {
						f[i][i + 1] = 2;
					}
				}

				for (int i = 0; i < n; i++) {
					int k = 1;
					while (i - k >= 0 && i + k < n && s[i - k] == s[i + k]) {
						f[i - k][i + k] = f[i - k + 1][i + k - 1] + 2;
						k++;
					}
				}

				for (int i = 0; i < n - 1; i++) {
					int k = 1;
					while (s[i] == s[i + 1] && i - k >= 0 && i + 1 + k < n && s[i - k] == s[i + 1 + k]) {
						f[i - k][i + 1 + k] = f[i - k + 1][i + 1 + k - 1] + 2;
						k++;
					}
				}

				int max = 0, I = 0;
				for (int i = 0; i < n; i++) {
					for (int j = i; j < n; j++) {
						if (max < f[i][j]) {
							max = f[i][j];
							I = i;
						}
					}
				}

				return s.substr(I, max);

			}
		};

* 22 Generate Parentheses
	\Problem
		Given n pairs of parentheses, write a function to generate all combinations of well-formed parentheses.
		Input: n = 3
		Output: ["((()))","(()())","(())()","()(())","()()()"]

	\Code{C++}
		class Solution {
		public:
			vector<string> generateParenthesis(int n) {
				vector<string> ans;
				
				string a = "";
				for(int j = 0;j<n;j++){
					a += "(";
				}
				
				ans.push_back(a);
				
				for(int i=0;i<n-1;i++){
					int m = ans.size();
					
					for(int j=0;j<m;j++){
						string b = ans[j].substr(0, i + (ans[j].length() - n) + 1);
						int kn = n + i + 1- ans[j].length();
						
						for(int k = 0;k< kn;k++){
							b += ')';
							ans.push_back(
								b + ans[j].substr(i + (ans[j].length() - n) + 1, ans[j].length() - (i + (ans[j].length() - n) + 1))
							);
						}
					}
				}
				
				int m = ans.size();

				for (int j = 0; j < m; j++) {
					for (int i = ans[j].length(); i < 2 * n; i++) {
						ans[j] += ')';
					}
				}

				for (int j = 0; j < m; j++) {
					for (int i = j + 1; i < m; i++) {

						if (ans[i] == ans[j]) {
							ans.erase(ans.begin() + i);
							i--; m--;
						}
					}
				}
				
				return ans;

			}
		};

* 23. Merge k Sorted Lists
	\Code{C++}
		class Solution {
		public:
			ListNode* mergeKLists(vector<ListNode*>& lists) {
				ListNode* root = new ListNode, *ptr = root;
				root->next = nullptr;

				int n = lists.size();

				while (1) {
					int minn = 0x7FFFFFFF, ind;

					int fg = 1;

					for (int i = 0; i < n; i++) {
						if (lists[i] != nullptr && minn > lists[i]->val) {
							minn = lists[i]->val;
							ind = i;
						}
						if (lists[i] != nullptr)
							fg = 0;
					}

					if (fg)
						break;

					ptr->next = lists[ind];
					lists[ind] = lists[ind]->next;
					ptr = ptr->next;
				}
				return root->next;
			}
		};

* 25. Reverse Nodes in k-Group
	\Code{C++}
		class Solution {
		public:
			void inv(ListNode*& st, ListNode*& ed) {
				if (st == ed)
					return;

				if (st->next == ed) {
					st->next = ed->next;
					ed->next = st;
					return;
				}

				ListNode
					* ptr = st,
					* ptr2 = st->next,
					* ptr3 = st->next->next,
					* mk = ed->next;

				while (ptr2 != mk) {
					ptr2->next = ptr;

					ptr = ptr2;
					ptr2 = ptr3;
					if (ptr3 != nullptr)
						ptr3 = ptr3->next;
				}
				st->next = mk;
			}

			void fun(ListNode*& st, ListNode*& ed, int k) {
				ed = st;
				while (--k) {
					if (ed == nullptr)
						return;
					ed = ed->next;
				}
			}

			ListNode* reverseKGroup(ListNode* head, int k) {

				int fg = 1;
				ListNode* st = head, * ed = head, * pre = nullptr;

				while (1) {
					fun(st, ed, k);

					if (ed == nullptr)
						return head;

					if (pre != nullptr)
						pre->next = ed;
					inv(st, ed);

					if (fg) { head = ed; fg = 0; }
					
					pre = st;
					st = st->next;
				}

				return head;
			}
		};

* 28 Implement strStr()
	\Code{C++}
		class Solution {
		public:
		int strStr(string s, string p) {
			int n = s.size(), m = p.size();
			if (m == 0) return 0;
			//设置哨兵
			s.insert(s.begin(), ' ');
			p.insert(p.begin(), ' ');
			vector<int> next(m + 1);

			//预处理next数组
			for (int i = 2; i <= m; i++) {
				int j = next[i - 1];

				while (j != 0 and p[i] != p[j + 1])
					j = next[j];

				if (p[i] == p[j + 1])
					j++;

				next[i] = j;
			}

			for (int i = 0; i <= m; i++) {
				cout << next[i] << ' ';
			}
			//匹配过程
			for (int i = 1, j = 0; i <= n; i++) {
				while (j != 0 and s[i] != p[j + 1])
					j = next[j];
				if (s[i] == p[j + 1])
					j++;
				if (j == m)
					return i - m;
			}
			return -1;
		}

		};

* 31. Next Permutation
	\Code{C++}
		class Solution {
		public:
			void nextPermutation(vector<int>& a) {
				int n = a.size();
		
				for (int i = n - 1; i >= 0; i--) {
					if(i == 0){
						inv(a, 0, n);
						break;
					}
					// find 1st
					else if (a[i - 1] < a[i]) {
						// inv last
						inv(a, i, n);
						
						int ind;
						for (int j = i; j < n; j++) {
							if (a[j] > a[i - 1]) {
								ind = j;
								break;
							}
						}
						swap(a[ind], a[i-1]);
						
						break;
					}
				}
			}
			
		void inv(vector<int>& a, int st, int ed) {
			for (int i = st; i < (st+ed) / 2; i++) {
				swap(a[i], a[ed + st - 1 - i]);
			}
		}
		
		};

* 39 Combination Sum
	\Problem
		(input) $\.a \in \mathbb Z_+^n, b \in \mathbb Z_+ \qu; a_i ≤ b$
		$
			\.x^T \.a = b
			\.x \in \mathbb N^n
		$
	
	\Algorithm

	\Code{C++}
		class Solution {
		public:
			int dot(vector<int>& a, vector<int>& b) {
				int n = a.size(), ans = 0;

				for (int i = 0; i < n; i++) {
					ans += a[i] * b[i];
				}

				return ans;
			}

			vector<vector<int>> combinationSum(vector<int>& candidates, int target) {
				vector<vector<int>> ans;
				vector<int> x;

				int n = candidates.size();
				for (int i = 0; i < n; i++) {
					x.push_back(0);
				}

				fun(0, x, candidates, target, ans);

				return ans;
			}

			void fun(int ind, vector<int>& x, vector<int>& a, int target, vector<vector<int>>& ans) {
				int n = a.size();

				if (ind == n - 1) {

					while (1) {
						int t = dot(x, a);

						for (int i = 0; i < x.size(); i++) {
							printf("%d ", x[i]);
						}printf(">%d \n", t);

						if (t > target) {
							x[ind] = 0;
							return;
						}

						else if (t == target) {
							getAns(x, a, ans);

							x[ind] = 0;
							return;
						}

						x[ind]++;
					}
				}

				while (1) {
					fun(ind + 1, x, a, target, ans);

					x[ind]++;

					int t = dot(x, a);
					if (t > target) {
						x[ind] = 0;
						return;
					}
				}

				x[ind] = 0;
				return;
			}

			void getAns(vector<int>& x, vector<int>& a, vector<vector<int>>& ans) {
				vector<int> t;
				int  n = a.size();

				for (int i = 0; i < n; i++) {
					for (int j = 0; j < x[i]; j++) {
						t.push_back(a[i]);
					}
				}
				ans.push_back(t);
			}
		};

* 41. First Missing Positive
	\Code{C++}
		class Solution {
		public:
				int firstMissingPositive(vector<int>& nums) {
						int n = nums.size();
						
						int a[n+5];
						memset(a, 0, sizeof(int) * (n+5));
						
						for(int i=0;i<n;i++){
								if(nums[i] > 0 && nums[i] < n+5){
										a[nums[i]] ++;
								}
						}
						
						for(int i=1;i<n+5;i++){
								if(a[i] == 0)
										return i;
						}
						return 0;
				}
		};

* 42
	\Problem

	\Code{C++}
		class Solution {
		public:
			int trap(vector<int>& height) {
				int n = height.size();

				int maxH = 0, maxHInd = 0, ans = 0;

				for (int i = 0; i < n; i++) {

					if (maxH <= height[i]) {

						for (int j = maxHInd; j < i; j++) {
							ans += maxH - height[j];
						}

						maxH = height[i];
						maxHInd = i;
					}

				}
				
				int maxH2 = 0, maxHInd2 = 0;
				
				for (int i = n-1; i >= maxHInd; i--) {
					
					if (maxH2 <= height[i]) {

						for (int j = maxHInd2; j > i; j--) {
							ans += maxH2 - height[j];
						}

						maxH2 = height[i];
						maxHInd2 = i;
					}
				}

				return ans;
			}
		};

* 44 
	\Problem
		(input) an string $s$ and a pattern $p$
		implement wildcard pattern matching with support for '?' and '*' where:
		- '?' Matches any single character.
		- '*' Matches any sequence of characters (including the empty sequence).
		The matching should cover the entire input string (not partial).

	\Property
		- $'**...*' <=> '*'$
		- 若 $p_y = '*'$, 则
			$f(x, y) = f(x, y-1) \vee f(x-1, y)$
			$
				f(x, y) = 1 & => f(x+k|_{k≥0}, y) = 1   \tag{'*'匹配任意长任意字符序列}
				f(x, y-1) = 1 & => f(x, y) = 1  \tag{'*'匹配空序列}
			$
			其中, $f(x, y)$: 字符串$s(1:x)$ 是否能被模式 $p(1:y)$匹配, 匹配为0, 不匹配为1.

			\Proof
				若 $p_y = '*'$
				$
					f(x, y-1) = 1 & => f(x, y) = 1  \tag{'*'匹配空序列}
					f(x-k|_{k≥1}, y-1) = 1 & => f(x-k|_{k≥1}, y) = 1  \tag{同上}
						& => f(x-1, y) = 1   \tag{'*'匹配任意长任意字符序列}
						& => f(x, y) = 1   \tag{同上}
				$
				当$f(x, y-1), f(x-1, y)$均为$0$时证明显然
				$ => f(x, y) = f(x, y-1) \vee f(x-1, y)$

	\Algorithm
		- 动态规划
			$
				f(x, y) &= \{\mb
					f(x - 1, y - 1)  \qu& s_x = p_y \ or\ p_y = '?'
					f(x, y-1) \vee f(x-1, y)  \qu& p_y = '*'
					0  \qu& other.
				\me\right.

				f(1, 1) &= \{\mb
					1  \qu& s_1 = p_1 \ or\ p_1 = '?'
					1  \qu& p_1 = '*'
					0  \qu& other.
				\me\right.  \tag{初始易知值}

				f(x|_{x>1}, 1) &= \{\mb
					1  \qu& p_1 = '*'
					0  \qu& other.
				\me\right.

				f(1, y|_{y>1}) &=  \{\mb
					1  \qu& p_1 = '*' \wedge p(2:y) \in \{'**', '*a', '*a*', '*?', '*?*'\}  &\qu; ('a' \Leftarrow '**', '*a', '*a*', '*?', '*?*')
					1  \qu& (s_1 = p_1 \vee p_1 = '?') \wedge (\wedge_{i=2}^y p_i = '*')  &\qu ('a' \Leftarrow 'a*')
					0  \qu& other.
				\me\right.
			$
			- $f(x, y)$: 字符串$s(1:x)$ 是否能被模式 $p(1:y)$匹配, 匹配为1, 不匹配为0.
			- 特例: $'' \Leftarrow '**...*'$

	\Code{C++}
		class Solution {
		public:
			bool isMatch(string s, string p) {
				int ns = s.length(), np = p.length();

				int is= 0, ip = 0, preS, preP, back = 0;

				while (is < ns) {
					if (ip < np && s[is] == p[ip] || p[ip] == '?') {
						is++;
						ip++;
						continue;
					}

					else  if (ip < np && p[ip] == '*') {
						while (ip < np && p[ip] == '*')
							ip++;

						if (ip == np)
							return true;

						back = 1;
						preS = is;
						preP = ip;
					} 
					else if (back) {
						is = ++preS;
						ip = preP;
					}

					else
						return false;
				}

				if (is == ns) {
					while (ip < np && p[ip] == '*')
						ip++;
				}

				if (is == ns && ip == np)
					return true;
				return false;
			}
		};

	\Code{C++}
		class Solution {
		public:
			bool isMatch(string s, string p) {
				int m = s.length(), 
					n = p.length();

				if (m == 0) {
					for (int j = 0; j < n; j++) {
						if (p[j] != '*') {
							return false;
						}
					}
					return true;
				}

				vector<vector<bool>> f(m, vector<bool>(n, false));

				// init
				if (s[0] == p[0] || p[0] == '?') {
					f[0][0] = 1;

					for (int j = 1; j < n; j++) {
						if (p[j] == '*') {
							f[0][j] = 1;
						}
						else break;
					}
				}
				else if (p[0] == '*') {
					for (int i = 0; i < m; i++) {
						f[i][0] = 1;
					}

					for (int j = 1; j < n; j++) {
						if (p[j] == '*') {
							f[0][j] = 1;
						}
						else if (s[0] == p[j] || p[j] == '?') {
							f[0][j] = 1;
							while (++j < n && p[j] == '*') {
								f[0][j] = 1;
							}
							break;
						}
						else break;
					}
				}

				else return false;

				// f
				for (int i = 1; i < m; i++) {
					for (int j = 1; j < n; j++) {
						if (s[i] == p[j] || p[j] == '?') {
							f[i][j] = f[i - 1][j - 1];
						}

						else if (p[j] == '*') {
							f[i][j] = f[i - 1][j] || f[i][j - 1];
						}
					}
				}

				for (int i = 0; i < m; i++) {
					for (int j = 0; j < n; j++) {
						printf("%d", f[i][j] ? 1 : 0);
					}printf("\n");
				}

				return f[m - 1][n - 1];
			}
		};

* 45 Jump Game II
	\Problem
		(input) $\.a \qu; a_i \in \mathbb N$
		$
			\min_{\.x \in \mathbb N^{1:(\dim{a}-1)}} \qu& \dim \.x
			s.t. \qu& x_i ≤ a(\sum_{k=0}^{i-1} x_k)
				& \.1^T \.x = \dim \.a - 1
		$

	\Algorithm
		- 动态规划
			$
				f(i) &= \min \{f(i-k) + 1 | k \in 1:\min(a_i, i), f(i-k) > 0 \ when\  i-k > 0\} 
				=> f(i+k) &\gets \min (f(i+k), f(i) + 1)  \qu; k \in 1:\min(a_i, \dim \.a - i)
				f(0) &= 0  \tag{初始易知值}
				f(\dim \.a) &= \dim \.x^*  \tag{答案}
			$

			- $f(k)$: 当$\.1^T \.x = i$时的优化问题的解, 无解时为0.

	\Code{C++}
		class Solution {
		public:
			int jump(vector<int>& nums) {
				int n = nums.size();
				int f[n];

				for (int i = 0; i < n; i++) {
					f[i] = 0xFFFFFFF;
				}
				f[0] = 0;

				for (int i = 0; i < n - 1; i++) {
					for (int k = 1; k <= nums[i]; k++) {
						if (i + k >= n)
							break;

						f[i + k] = min(f[i + k], f[i] + 1);
					}
				}

				return f[n - 1];
			}
		};


* 51. N-Queens
	\Problem
		(input) $n \in \mayhbb Z^+$
		
	\Code{C++}
		class Solution {
		public:
			vector<vector<string>> Ans;

			vector<vector<string>> solveNQueens(int n) {

				int queen[n];
				for (int i = 0; i < n; i++) 
					queen[i] = i;

				Permutation(queen, 0, n);
				return Ans;
			}

			void Permutation(int* arr, int cur, int N) {
				if (!judge(arr, cur - 1)) 
					return;
				if (cur == N - 1 && judge(arr, N - 1)) {
					noteAns(arr, N);
					return; 
				}

				for (int i = cur; i < N; i++) {
					swap(arr[cur], arr[i]);
					Permutation(arr, cur + 1, N);
					swap(arr[cur], arr[i]);
				}
			}

			bool judge(int* arr, int cur) {
				for (int i = 0; i < cur; i++)
					if ((arr[cur] + cur == arr[i] + i) ||
						(arr[cur] - cur == arr[i] - i)) 
						return false;
				return true;
			}

			void noteAns(int* arr, int N) {
				vector<string> v;
				for (int i = 0; i < N; i++) {
					string s;

					for (int k = 0; k < N; k++)
						s += '.';
					s[arr[i]] = 'Q';
					v.push_back(s);
				}
				Ans.push_back(v);
			}

		};

* 52. N-Queens II
	\Code{C++}
		class Solution {
		public:
			int Ans = 0;
		
			int totalNQueens(int n) {
		
				int queen[n];
				for (int i = 0; i < n; i++)
					queen[i] = i;
		
				Permutation(queen, 0, n);
				return Ans;
			}
		
			void Permutation(int* arr, int cur, int N)
			{
				if (!judge(arr, cur - 1))
					return;
				if (cur == N - 1 && judge(arr, N - 1)) {
					Ans++;
					return;
				}
		
				for (int i = cur; i < N; i++) {
					swap(arr[cur], arr[i]);
					Permutation(arr, cur + 1, N);
					swap(arr[cur], arr[i]);
				}
			}
		
			bool judge(int* arr, int cur) {
				for (int i = 0; i < cur; i++)
					if ((arr[cur] + cur == arr[i] + i) ||
						(arr[cur] - cur == arr[i] - i))
						return false;
				return true;
			}
		
		};

* 54 
	\Code{C++}
		class Solution {
		public:
			vector<int> spiralOrder(vector<vector<int>>& matrix) {
				vector<int> ans;
				
				int m = matrix.size(), n = matrix[0].size();
				
				int x = 0, y = 0, dimX = 0, dimY = 1;
				
				for(int i=0;i<m*n;i++){
					ans.push_back(matrix[x][y]);
					matrix[x][y] = 0xFFFFFFF;
					
					if(dimY > 0 && (y + dimY == n || matrix[x][y + dimY] == 0xFFFFFFF)){
						dimX = 1;
						dimY = 0;
					}
					
					else if(dimY < 0 && (y + dimY == -1 || matrix[x][y + dimY] == 0xFFFFFFF)){
						dimX = -1;
						dimY = 0;
					}
					
					else if(dimX > 0 && (x + dimX == m || matrix[x + dimX][y] == 0xFFFFFFF)){
						dimX = 0;
						dimY = -1;
					}
					
					else if(dimX < 0 && (x + dimX == -1 || matrix[x + dimX][y] == 0xFFFFFFF)){
						dimX = 0;
						dimY = 1;
					}
					
					x += dimX;
					y += dimY;
				}
				
				return ans;
			}
		};

* 60. Permutation Sequence
	\Code{C++}
		class Solution {
		public:
			int T = 0;
			bool F = 0;
			string Ans;
			
			string getPermutation(int n, int k) {
				T = k;
				
				int a[n];
				for(int i=0;i<n;i++)
					a[i] = i+1;
				
				for(int i=0;i<k-1;i++)
					std::next_permutation(a,a+n);
				
				//Permutation(a, 0, n);
				for(int i=0;i<n;i++)
					Ans+='0' + a[i];
				
				return Ans;
			}
			
			void Permutation(int* a, int cur, int N) {
				if(F)
					return;
				if(cur == N-1){
					T--;
					for(int i=0;i<N;i++)
						printf("%d ", a[i]); printf("\n");
					if(T == 0){
						for(int i=0;i<N;i++)
							Ans+='0' + a[i];
						F = 1;
						return;
					}
				}
		
				for (int i = cur; i < N; i++) {
					swap(a[cur], a[i]);
					Permutation(a, cur + 1, N);
					if(F)
						return;
					swap(a[cur], a[i]);
					
				}
			}
		};

* 62 Unique Paths
	\Problem
		(input) $x_0, y_0 \in \mathbb N$
		方块图,$(0, 0) \to (x_0, y_0)$,只能走右、下的所有路径数.

	\Algorithm
		- 动态规划
		$
			f(x, y) &= f(x-1, y) + f(x, y-1)
			f(0, y) &= 1  \tag{初始易知值}
			f(x, 0) &= 1
		$

		- $f(x, y)$: 位置$x, y$处的路径数.

	\Code{C++}
		class Solution {
		public:
			int uniquePaths(int m, int n) {
				int f[m][n];
				
				for(int i = 0;i<m;i++){
					f[i][0] = 1;
				}
				
				for(int i = 0;i<n;i++){
					f[0][i] = 1;
				}
				
				for(int i = 1;i<m;i++){
					for(int j=1;j<n;j++){
						f[i][j] = f[i][j-1] + f[i-1][j];
					}
				}
				
				return f[m-1][n-1];
			}
		};

* 65. Valid Number
	\Code{C++}
		class Solution {
		public:
		bool isNumber(string s) {
			int n = s.length(), ind = 0;
		
			int Fg = 0;
			
			if (s[n-1]  == 'e' || s[n-1] == 'E')
				return false;
		
			if (s[ind] == '+' || s[ind] == '-')
				ind++;
		
			while (ind < n) {
				while (s[ind] >= '0' && s[ind] <= '9')
					ind++;
		
				if (s[ind] == 'e' || s[ind] == 'E') {
					if (ind == 0 || (ind == 1 && (s[0] == '+' || s[0] == '-')) || Fg == 1)
						break;
		
					Fg = 1;
		
					ind++;
		
					if (s[ind] == '+' || s[ind] == '-'){
						if(ind == n-1)
							return false;
						ind++;
							
					}
				}
		
				else if (s[ind] == '.') {
					if (Fg == 1 || Fg == 2)
						break;
					
					char a = 0, b = 0;
					if(ind-1>=0) a = s[ind-1];
					if(ind+1< n) b = s[ind+1];
					
					if(!((a >= '0' && a <= '9') || (b>= '0' && b <= '9')))
						break;
		
					Fg = 2;
		
					ind++;
				}
		
				else
					break;
			}
		
			if (ind == n)
				return true;
			return false;
		}
		
		};
		